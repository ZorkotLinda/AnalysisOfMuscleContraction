\documentclass[12pt, a4paper]{article}                                   

\usepackage[T2A]{fontenc}             
\usepackage[utf8]{inputenc}                                             
\usepackage[ukrainian]{babel}   
\usepackage{geometry}
\geometry{%% page margins                                              
    a4paper,                                                            
    left=15mm,                                                          
    right=15mm,                                                         
    top=10mm,                                                           
    bottom=20mm,                                   
    }                                                                       

\title{Аналіз експериментальних данних впливу окситоцину та хелеретрину на
        скорочення міометрію}
\date{}


\begin{document}
\maketitle

\thispagestyle{empty}
\section{Візуалізація та вирівнювання даних}
\textbf{Слайд 1 - візуалізація даних до і після вирівнювання}
\\
Перші 20 хвилин запису -- артефакти, тому ці дані були вирізані і не враховані
при вирівнюванні та подальших розрахунках. Базова лінія ідентифікувалася шляхом
фітування поліноміалу, після чого значення базової лінії віднімалися.

\section{Автоматизований пошук піків}
\textbf{Слайд 2 - повне вирівнювання та знайдені піки}
\\
\textbf{Слайд 3 - червоними лініями зображено ширину піків. Початок і кінець
лінії --  межі інтегрування}
\\
\textbf{Слайд 4 - червоними лініями зображено напівширину піків.}
\\
Пошук місцезнаходження та ширини піків відбувався по повністю вирівняним даним. Мінімальна
висота піків визначалася як 0.2 від максимальної. Відстань між піками -- 200
точок. Значення амплітуди піків бралися з нормальних даних. 

\section{Розрахунок площі, розрахунок середніх значень та стандартної похибки.}
\textbf{Слайд 5 - стовпчикові діаграми середніх значень амплітуди скорочень,
площі під піками та напівширини піків за дії різних чинників та у контролі, а
також відповідні стандартні похибки середнього} 

\section{Перевірка статистичної гіпотези}
Дані перевірялися на нормальність розподілу за допомогою теста Шапіро-Уілка,
після чого у залежності від результату проводили або попарний Т-тест Стьюдента,
або U-тест Манна-Уітні.

\end{document}




