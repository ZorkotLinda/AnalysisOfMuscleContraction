\documentclass[12pt, a4paper]{article}                                   

\usepackage[T2A]{fontenc}             
\usepackage[utf8]{inputenc}                                             
\usepackage[ukrainian]{babel}   
\usepackage{geometry}
\geometry{%% page margins                                              
    a4paper,                                                            
    left=15mm,                                                          
    right=15mm,                                                         
    top=30mm,                                                           
    bottom=20mm,                                   
    }                                                                       
\usepackage{csvsimple}   

\title{Аналіз експериментальних даних впливу окситоцину та хелеретрину на
        скорочення міометрію}
\date{}


\begin{document}
\maketitle

\thispagestyle{empty}
\section{Візуалізація та вирівнювання даних}
\textbf{Слайд 1 - візуалізація даних до і після вирівнювання}
\\
Перші 20 хвилин запису -- артефакти, тому ці дані були вирізані і не враховані
при вирівнюванні та подальших розрахунках. Базова лінія ідентифікувалася шляхом
фітування поліноміалу, після чого значення базової лінії віднімалися.

\section{Автоматизований пошук піків}
\textbf{Слайд 2 - повне вирівнювання та знайдені піки}
\\
\textbf{Слайд 3 - червоними лініями зображено ширину піків. Початок і кінець
лінії --  межі інтегрування}
\\
\textbf{Слайд 4 - червоними лініями зображено напівширину піків.}
\\
Пошук місцезнаходження та ширини піків відбувався по повністю вирівняним даним. Мінімальна
висота піків визначалася як 0.2 від максимальної. Відстань між піками -- 200
точок. Значення амплітуди піків бралися з нормальних даних. 

\section{Розрахунок площі, розрахунок середніх значень та стандартної похибки.}
\textbf{Слайд 5 - стовпчикові діаграми середніх значень амплітуди скорочень,
площі під піками та напівширини піків за дії різних чинників та у контролі, а
також відповідні стандартні похибки середнього} 

\section{Перевірка статистичної гіпотези}
Дані перевірялися на нормальність розподілу за допомогою теста Шапіро-Уілка,
після чого у залежності від результату проводили або попарний Т-тест Стьюдента,
або U-тест Манна-Уітні.

\newpage
\thispagestyle{empty}
\section{Висновки\\}
\begin{center}
\csvautotabular{Stats/t_tests_pvals.csv}
\end{center}
\par
Вище можна бачити таблицю зі значеннями статистичної значимості різниці в середніх амплітудах
для різних груп. Для всіх пар була знайдена статистично значима різниця, крім пари окситоцин і
окситоцин + хелеретрин.

По отриманих ерорбарах видно, що сам по собі хелеретрин пригнічує скорочення міометрію (зменшення
амплітуди, площі), а окситоцин підсилює скорочення міометрію (збільшення амплітуди, площі).
Пов'язано це з тим, що хелеретрин є інгібітором протеїнкінази С, а окситоцин
збільшує внутрішньоклітинну концентрацію $Ca^{2+}$ та активує протеїнкіназу С. Челеретрин здатний проходити через
цитоплазматичну мембрану і безпосередньо інгібувати
протеїнкіназу С, таке інгібування відбувається досить швидко. Окситоцин діє на протеїнкіназу
через G-білок: дисоціація $\alpha$-субодиниці, активація фосфоліпази С, розщеплення PIP$_2$ на
DAG та IP$_3$, IP$_3$ відкриває кальцієві канали на ЕПР, виходить $Ca^{2+}$, кальцій зв'язується
з кальмодуліном і разом з DAG цей комплекс активує PKC. За раухнок того, що в цьому каскаді
багато ланок, окситоцин підвищує амплітуду скорочення з певною часовою затримкою.

PKC підвищує амлітуду скорочення за рахунок того, що вона фосфорилює проміжні філаменти, білок
кальдесмон, який в дефосфорильованому стані інгібує зв'язування міозину з актином, інші
актинзв'язуючі білки, фосфорилює і активує білок CPI-17, який інгібує міозин
фосфатазу і таким чином не дає їй дефосфорилювати легкі ланцюгі міозину.
                                                                                
Але виникає питання, чому сумарна дія окситоцину і хелеретрину не особливо
відрізняється від дії окситоцину окремо. По-перше, можна сказати, що протеїнкіназа С грає не
першочергову роль у окситоцин-індукованому скороченні гладеньких м'язів, а,
скоріше, доповнюючу, оскільки першочерговий вплив на скорочення здійснить
відкриття кальцієвих каналів СПР.
\par
Можливе ще таке пояснення: якусь частину
протеїнкінази С окситоцин все таки активує і відбувається "interplay"\ між ефектом інгібування
хелеретрину і активацією окситоцину, хоча інгібування хелеретрином мало б
відбутися швидше. Також відомо, що хелеретрин інгібує $Ca^{2+}$-АТФази на
мембрані CПР і плазматичній мембрані, при чому у випадку SERCA ефективна концентрація інгібування та ж
сама, що і для протеїнкінази С. А тому за рахунок того, що окситоцин підвищує, за описаним
вище механізмом, концентрацію $Ca^{2+}$, а хелеретрин зменшує виведення $Ca^{2+}$ з цитозолю,
спостерігається підвищення амплітуди скорочення (за рахунок активності кінази
легких ланцюгів міозину). Таким чином, ці дві речовини набувають
взаємодоповнюючої дії.

\end{document}




